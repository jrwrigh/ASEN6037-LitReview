\documentclass{ucb}
\usepackage{cleveref}

\begin{document}
\ucbcourse{\textbf{ASEN 6037} Turbulent Flows}
\ucbtitle{Literature Review: \textit{A Hybrid {{RANS}}-{{LES}} Approach with Delayed-{{DES}} and Wall-Modelled {{LES}} Capabilities}}
\ucbauthors{James Wright \\}
\ucblocation{Boulder, Colorado}
\ucbcover{}

\section{Introduction}
This literature review will cover the content and background of \citetitle{shurHybridRANSLESApproach2008}\cite{shurHybridRANSLESApproach2008}. 

\section{Overview of Prior Work}
This paper introduces a new hybrid RANS-LES turbulence model, later known as the improved delayed detached eddy simulations (IDDES). Before reviewing the paper, let's review the history of hybrid turbulence models in general.

IDDES can trace it's main roots back to the seminal detached eddy simulation (DES)~\cite{SpalartP.R.JouW-H.StreletsM.Allamaras1997} model, first proposed by \citeauthor{SpalartP.R.JouW-H.StreletsM.Allamaras1997}. The concept and motivation of hybrid models is fairly simple. LES is very expensive (particularly near the wall) for high Re flows, while RANS is significantly cheaper but not reliably accurate for many types of flow problems. Hybrid models combine the two to exploit their strengths and cover up their weakness. In general, RANS is responsible for flow regions that are very expensive for LES (namely near the wall) and LES is responsible for regions where RANS is not adequate. This is visually shown in \cref{fig:HRLMS}, where the shaded regions represent the parts of the turbulent spectrum that are modeled.

\begin{figure}[h]
    \centering
    \includegraphics[]{img/TurbulenceModelsResolution_Tucker2015.jpg}
    \caption{Turbulence model hierarchy. Reproduced from~\cite{Tucker2015} under the Creative Commons Attribution License (\url{https://creativecommons.org/licenses/by/4.0/}).}
    \label{fig:HRLMS}
\end{figure}

% While there are some conceptual issues with the original DES formulation, %! reference to desire for WMLES instead of RANS near wall
The transition between the two models is one of the primary achilles's heel of DES-like methods. This is true both in determining where transition should occur and in transforming the modeled stress in the RANS region into resolved turbulence structures required by LES. The transition point was the primary reason for the first (of many) modifications to the original DES model. The DES model based the transition point on the grid size. This lead to issues (Modeled Stress Depletion and consequently Grid Induced Separation) when the model was used on different sized near-wall grids. This issue was addressed by \citeauthor{Menter2002} in \citedate{Menter2002} by introducing a shielding function to the DES formulation~\cite{Menter2002}. The original transition mechanism in DES was suppressed inside a boundary layer, as determined by the shielding function. This idea was originally only implemented with the SST-based DES model, but was generalized into the delayed detached eddy simulation (DDES) by \citeauthor{Spalart2006} in \citedate{Spalart2006}~\cite{Spalart2006}.

There have been attempts to use DES as a wall-modeled LES (WMLES), where RANS is used for modeling turbulence in a thin region within the boundary layer. However, at the transition between the RANS and LES regions, there exists what is known as the log-layer mismatch, referring to the log-law portion of the boundary layer. The issue is that while both RANS and LES both accurately predict the logarithmic relationship between \(u^+\) and \(y^+\), they predict them at different magnitudes. Thus, at the boundary between then, the velocity profile must transition from the log-law relationship predicted by RANS to the one predicted by LES.

\section{Work Summary}
This work introduces a new hybrid RANS-LES turbulence model dubbed the improved delayed detached eddy simulation (IDDES). The goal of the model is to allow for WMLES and standard DDES operation in the same model. Additionally, it aims to resolve the log-layer mismatch seen in other applications of DES-esque models for WMLES.
%TODO Add sentence regarding things to be reviewed. ie:
% To achieve this, a new subgrid length-scale was developed, in addition to the chagnes to the model euqations themselves.

\subsection{Subgrid Length-Scale}
The first major part introduced in the paper is a new definition for the subgrid length-scale, \(\Delta \). Traditionally, this is approximated by either the cube root of the element/cell volume or by the maximum of three element/cell spacings. Most SGS models use \(\Delta \) as a significant input to their calculation. However, the problem with either of these definitions is that the model coefficients for SGS model in near-wall flows is significantly different for free stream flows. Ideally, the SGS model coefficients should constant regardless of their location in the flow. To achieve this, \citeauthor{shurHybridRANSLESApproach2008} created a new definition of \(\Delta \) that used wall distance as an input. Far from the wall, \(\Delta \) should be equal to the maximum of the element/cell spacing. Close to the wall, \(\Delta \) should be made some function of the wall-parallel spacings. Ignoring the wall-normal spacing is done to avoid the sharp changes in the wall-normal spacing common with meshes close to the wall. To transition between these two extremes, it is assumed that \(\Delta \) is a linear function of wall distance. Lastly, it is also noted that \(\Delta \) should be constrained to be between the minimum and maximum wall spacings. The end result is

\begin{equation}\label{eq:delta}
    \Delta = \min \left \{ 
        \max \left \{ C_w d_w, C_w h_{\max}, h_{wn}\right \} 
        , h_{\max} \right \}
\end{equation}

%TODO define these variables or use them in the above paragraph
where \(C_w\) is a constant, which was found to be 0.15. 
The two features of the above definition are that \(\Delta \) is reduced in the near-wall region and transitions to the "free-stream" value quickly. 

%TODO come back and add the actual results



\ucbbib{}
    
\end{document}